%!TEX root = paper.tex

%following two lines added to supress/fix the following error - amsthm.sty:444: LaTeX Error: Command \proof already defined.
\let\proof\relax
\let\endproof\relax

%following five lines added to supress/fix the following errors -
%algorithm2e.sty:1616: Too many }'s. [    }]
%algorithm2e.sty:1617: Extra \fi. [\fi]

\makeatletter
\newif\if@restonecol
\makeatother
\let\algorithm\relax
\let\endalgorithm\relax

\usepackage{paralist}
\usepackage{color}
\usepackage[colorlinks=true,pdfpagelabels=false,bookmarks=false]{hyperref}
\usepackage{cite}
\usepackage{graphicx}
\usepackage{ifthen}
\usepackage[ruled,vlined]{algorithm2e}
% \usepackage{flushend}
\usepackage{multirow}
\usepackage{xspace}
\usepackage{algorithmic}
\usepackage{amsfonts}
\usepackage{amsthm}
\usepackage{subfig}
\usepackage{expdlist}
% \usepackage{natbib}
\usepackage{tabulary}
\usepackage{framed}
\renewenvironment{leftbar}[1][\hsize]
{%
    \def\FrameCommand
    {%
        {\vrule width 1pt}%
        \hspace{5pt}%
        \fboxsep=\FrameSep%
    }%
    \MakeFramed{\hsize#1\advance\hsize-\width\FrameRestore}%
}
{\endMakeFramed}


\newcommand{\ie}{\textit{i.e.,}\xspace}
\newcommand{\eg}{\textit{e.g.,}\xspace}
\newcommand{\etal}{\textit{et al.}\xspace}
\newcommand{\nb}{\textit{N.B.,}\xspace}

% \def\tech{\textsc{Brain Visualization}\xspace}
% \def\tech{\textsc{The Brain}\xspace}
\def\tech{\textsc{Cerebro}\xspace}
\def\jinsight{\textsc{Jinsight}\xspace}
\def\extravis{\textsc{Extravis}\xspace}
\def\stars{\textsc{Constellation} visualization\xspace}
\def\jive{\textsc{Jive}\xspace}
\def\synctrace{\textsc{SyncTrace}\xspace}
\def\tracediff{\textsc{TraceDiff}\xspace}
\def\seesoft{\textsc{SeeSoft}\xspace}


\hyphenation{fault-lo-cal-iz-a-tion}
\hyphenation{fault-di-ag-no-sis}
\hyphenation{Ja-va}


\newtheoremstyle{mydef}% ⟨name⟩ 
{3pt}% ⟨Space above⟩ 
{3pt}% ⟨Space below⟩
% {\addtolength{\leftskip}{1em}\addtolength{\rightskip}{1em}\slshape} % ⟨Body font⟩
{\slshape} % ⟨Body font⟩
{}% ⟨Indent amount⟩
{\bfseries}% ⟨Theorem head font⟩
{.}% ⟨Punctuation after theorem head⟩
{1em}% ⟨Space after theorem head⟩2
{\thmname{#1}\thmnumber{ #2}:\textit{\thmnote{ #3}}}% ⟨Theorem head spec (can be left empty, meaning ‘normal’)⟩

\theoremstyle{mydef}
\newtheorem{definition}{Definition}
\newtheorem{lemma}{Lemma}
\newtheorem{theorem}{Theorem}

% JIM: I'm defining this here as we seem to have different styles on how to handle such unofficial, pseudo ``subsections''. I like to use something like these pseudo subsections, too, especially for this IEEE format in which subsections, and even worse subsubsections, look bad.
% Harry's style:
% \newcommand{\mysubsection}[1]{\textit{\textbf{#1}}~~~}
% Jim's style: (Jim likes Jim's style better. Big surprise! ;-)
% \newcommand{\mysubsection}[2]{\vspace{1.25mm}\noindent\textit{\textbf{#1.} #2}}
% \newcommand{\mysubsubsection}[1]{\vspace{1.25mm}\noindent\textit{#1.}}
\newcommand{\mysubsection}[1]{\vspace{1.25mm}\noindent\textit{\textbf{#1.}}}
\newcommand{\mysubsubsection}[1]{\vspace{1.25mm}\noindent\textit{#1.}}


%%%%%%%%%%%%%%%%%%%%%%%%%%%%%%%%%%%%%%%%%%%%%%%%%%%%%%%%%%%%%%%%%
%% The following definitions are to extend the LaTeX algorithmic 
%% package with SWITCH statements and one-line structures.
%% The extension is by 
%%   Prof. Farn Wang 
%%   Dept. of Electrical Engineering, 
%%   National Taiwan University. 
%% 
\newcommand{\SWITCH}[1]{\STATE \textbf{switch} (#1)}
\newcommand{\ENDSWITCH}{\STATE \textbf{end switch}}
\newcommand{\CASE}[1]{\STATE \textbf{case} #1\textbf{:} \begin{ALC@g}}
\newcommand{\ENDCASE}{\end{ALC@g}}
\newcommand{\CASELINE}[1]{\STATE \textbf{case} #1\textbf{:} }
\newcommand{\DEFAULT}{\STATE \textbf{default:} \begin{ALC@g}}
\newcommand{\ENDDEFAULT}{\end{ALC@g}}
\newcommand{\DEFAULTLINE}[1]{\STATE \textbf{default:} }
%% 
%% End of the LaTeX algorithmic package extension.
%%%%%%%%%%%%%%%%%%%%%%%%%%%%%%%%%%%%%%%%%%%%%%%%%%%%%%%%%%%%%%%%%

\def\d3js{\textsc{D3.js}\xspace}
\def\blinky{\textsc{Blinky}\xspace}
\def\mantis{\textsc{Mantis}\xspace}
\def\tarantula{\textsc{Tarantula}\xspace}
\def\aspectj{\textsc{AspectJ}\xspace}
\def\tetris{\textsc{Tetris}\xspace}
\def\nanoxml{\textsc{NanoXML}\xspace}
\def\javac{\textsc{Javac}\xspace}
\def\jpacman{\textsc{JPacman}\xspace}
\def\jedit{\textsc{jEdit}\xspace}
\def\ibugs{\textsc{iBugs}\xspace}


\def\denseitems{
  \itemsep1pt plus1pt minus1pt
  \parsep0pt plus0pt
  \parskip0pt
  \topsep0pt
}

%% Alternative to itemize
\newenvironment{smallitemize}{
   \setlength{\topsep}{0pt}
   \setlength{\partopsep}{0pt}
   \setlength{\parskip}{0pt}
   \begin{itemize}
   \setlength{\leftmargin}{.2in}
   \setlength{\parsep}{0pt}
   \setlength{\parskip}{0pt}
   \setlength{\itemsep}{0pt}}{\end{itemize}}

%% Alternative to enumerate
\newenvironment{smallenumerate}{
   \setlength{\topsep}{0pt}
   \setlength{\partopsep}{0pt}
   \setlength{\parskip}{0pt}
   \begin{enumerate}
   \setlength{\leftmargin}{.2in}
   \setlength{\parsep}{0pt}
   \setlength{\parskip}{0pt}
   \setlength{\itemsep}{0pt}}{\end{enumerate}}

%% Alternative to description
\newenvironment{smalldescription}{
   \setlength{\topsep}{0pt}
   \setlength{\partopsep}{0pt}
   \setlength{\parskip}{0pt}
   \begin{description}
   \setlength{\leftmargin}{.2in}
   \setlength{\parsep}{0pt}
   \setlength{\parskip}{0pt}
   \setlength{\itemsep}{0pt}}{\end{description}}

% Set author comment colors
\newcommand{\nick}[1]
{
   {\color{red}\bf [#1]$_{\scriptscriptstyle\textit{nicholas}}$}
}
\definecolor{grayblue}{rgb}{0.4,0.49,0.59}
\newcommand{\jim}[1]
{
   {\color{grayblue}\bf [#1]$_{\scriptscriptstyle\textit{jim}}$}
}
\newcommand{\paco}[1]
{
   {\color{green}\bf [#1]$_{\scriptscriptstyle\textit{paco}}$}
}
\definecolor{darkgreen}{rgb}{0.2,0.55,0.1} 
\newcommand{\vijay}[1]
{
   {\color{darkgreen}\bf [#1]$_{\scriptscriptstyle\textit{vijay}}$}
}
\definecolor{violet}{rgb}{0.54,0.17,0.88} 
\newcommand{\jy}[1]
{
   {\color{violet}\bf [#1]$_{\scriptscriptstyle\textit{harry}}$}
}

% Allow author comments to be turned off
\newcommand{\includeAuthorComments}[1]
{
   \ifthenelse{\equal{#1}{0}}
   {
      \renewcommand{\nick}[1]
      {
         {} % empty definition
      }
      \renewcommand{\jim}[1]
      {
         {} % empty definition
      }
      \renewcommand{\paco}[1]
      {
         {} % empty definition
      }
      \renewcommand{\vijay}[1]
      {
         {} % empty definition
      }
      \renewcommand{\harry}[1]
      {
         {} % empty definition
      }
   }{}
}

% Jim's tweaks for making the PDF easier on the eyes when proofreading on screen.
% defaults:
\hypersetup{linkcolor=black,citecolor=black,urlcolor=black}
\pagecolor{white}
\color{black}

\newcommand{\pdfColorScheme}[1]
{
   \ifthenelse{\equal{#1}{0}}{ % Standard black text on white background
      \definecolor{foregc}{RGB}{0,0,0}
      \definecolor{backgc}{RGB}{255,255,255}
   }{}
   \ifthenelse{\equal{#1}{1}}{ % Inverted. White text on black background
      \definecolor{foregc}{RGB}{255,255,255}
      \definecolor{backgc}{RGB}{0,0,0}
   }{}
   \ifthenelse{\equal{#1}{2}}{ % Solarized light
      \definecolor{foregc}{RGB}{101,123,131}
      \definecolor{backgc}{RGB}{253,246,227}
   }{}
   \ifthenelse{\equal{#1}{3}}{ % Solarized dark
      \definecolor{foregc}{RGB}{131,148,150}
      \definecolor{backgc}{RGB}{0,43,54}
   }{}
   \ifthenelse{\equal{#1}{4}}{ % Monokai dark
      \definecolor{foregc}{RGB}{248,248,242}
      \definecolor{backgc}{RGB}{39,40,34}
   }{}

   \hypersetup{linkcolor=foregc,citecolor=foregc,urlcolor=foregc}
   \pagecolor{backgc}
   \color{foregc}
}


%% Alter some LaTeX defaults for better treatment of figures:
%    % See p.105 of "TeX Unbound" for suggested values.
%    % See pp. 199-200 of Lamport's "LaTeX" book for details.
%    %   General parameters, for ALL pages:
%    \renewcommand{\topfraction}{0.9}	% max fraction of floats at top
%    \renewcommand{\bottomfraction}{0.8}	% max fraction of floats at bottom
%    %   Parameters for TEXT pages (not float pages):
%    \setcounter{topnumber}{2}
%    \setcounter{bottomnumber}{2}
%    \setcounter{totalnumber}{4}     % 2 may work better
%    \setcounter{dbltopnumber}{2}    % for 2-column pages
%    \renewcommand{\dbltopfraction}{0.9}	% fit big float above 2-col. text
%    \renewcommand{\textfraction}{0.07}	% allow minimal text w. figs
%    %   Parameters for FLOAT pages (not text pages):
%    \renewcommand{\floatpagefraction}{0.7}	% require fuller float pages
%	% N.B.: floatpagefraction MUST be less than topfraction !!
%    \renewcommand{\dblfloatpagefraction}{0.7}	% require fuller float pages

%%%%makes a page be able to be filled with more figures
\renewcommand{\topfraction}{1.0}
\renewcommand{\bottomfraction}{1.0}
\renewcommand{\textfraction}{0}
\addtolength{\itemsep}{0pt}
